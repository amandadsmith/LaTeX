% `nomenclature.tex', an example set of nomenclature for a thermal engineering paper.
%
% Amanda D. Smith
% amanda.d.smith@utah.edu
%
% Created: 2017-07-12
%

\documentclass[]{article}
\pagenumbering{gobble}

\begin{document}

\section*{Nomenclature}

\begin{tabbing}
  XXXXXXX \= \kill% this line sets tab stop
  $A$ \> Area \\
  $C$ \> Fluid capacity\\
  $C_R$ \>Fluid capacity ratio  (or capacity rate ratio)\\  
  $c_p$ \> Specific heat at constant pressure  (per unit mass)\\
  $c_v$ \> Specific heat at constant volume  (per unit mass)\\
  $D$ \> Mass diffusivity\\
  $GFSSP$ \> Generalized Fluid System Simulation Program \\
  $h$ \> Enthalpy (per unit mass)\\
  $K_m$ \> Mass transfer coefficient\\
  $k$ \> Thermal conductivity\\

  $HVAC$ \> Heating, ventilation, and air conditioning \\
  $\dot{m}$ \> Mass flow rate \\
  $N$ \> Number of increments used for discretization\\
  $NTU$ \> Number of transfer units \\
  $ORC$ \> Organic Rankine cycle \\
  $Q$ \> Energy transferred as heat \\
  $\dot{Q}$ \> Rate of heat transfer \\
  $P$ \> Pressure \\
  $S$ \> Entropy \\
  $T$ \> Temperature ($T_{db}$ unless otherwise noted) \\
  $U$ \> Overall heat transfer coefficient \\
  $V$ \> Volume \\
  $W$ \> Work \\
  $\dot{W}$ \> Power\\
  
  $\gamma$ \> Ratio of specific heats, $c_p/c_v$ \\
  $\epsilon$ \> Effectiveness \\
  $\eta$ \> Efficiency \\  
  $\omega$ \> Humidity ratio\\\\[5pt]
  
  \textbf{\textit{Subscripts}}\\
  $act$ \> Actual \\
  $da$ \> Dry air \\
  $db$ \> Dry bulb \\
    $f$ \> Liquid water (saturated) \\
    $g$ \> Water vapor (saturated) \\
  $in$ \> Inlet conditions \\
  $lat$ \> Latent \\
  $ma$ \> Moist air \\
  $out$ \> Outlet conditions \\
  $s$ \> Isentropic \\
  $sat$ \> Saturation conditions \\
  $sens$ \> Sensible \\
  $wb$ \> Wet bulb %\\

 \end{tabbing}


\end{document}
